\documentclass[global,twocolumn]{svjour}
%\documentclass[global,twocolumn,referee]{svjour}
% Remove option referee for final version
%
% Remove any % below to load the required packages
%\usepackage{latexsym}
\usepackage{graphics}
\usepackage{graphicx}

\usepackage{amsmath}
% etc
%
% Insert the name of "your" journal with the command below:
\journalname{Applied Physics B}
%
\begin{document}
	%
\title{Silicon isotope separation in a SiF$_{4}$ molecular jet by two-frequency IR multiphoton dissociation}
	%\subtitle{Do you have a subtitle?\\ If so, write it here}
	\author{M Risaro\inst{1} \and V D'Accurso\inst{1} \and J Codnia\inst{1}% etc
		% \thanks is optional - remove next line if not needed
		\thanks{\emph{Present address:} Insert the address here if needed}%
	}                     % Do not remove
	%
	\offprints{}          % Insert a name or remove this line
	%
	\institute{DEILAP-CITEDEF-CONICET}
	%
	\date{Received: date / Revised version: date}
	% The correct dates will be entered by the editor
	%
	\maketitle
	%
	\begin{abstract}
		Silicon isotope separation has been performed by infrared multiphoton dissociation of pure SiF$_{4}$ molecular jet, using two TEA CO$_{2}$ laser. The dissociation process was monitored with a Time-of-Flight mass spectrometer by UV multiphoton ionization, using the fourth harmonic of a pulsed Nd:YAG laser. The dissociation yield and enrichment factor has been studied in terms of the lasers fluence, wavenumber and delay time. The results shows a remarkable increase in the dissociation yield and enrichment factor in the two-frequency technique compare with the single-frequency IRMDP.
	\end{abstract}
	%
\section{Introduction}
\label{intro}
Why silicon isotope enrichment. Different applications of SiF$_{4}$

Laser isotope separation based on infrared multiphoton dissociation

Improvement of two frequency IRMPD in different molecules. Application in silicon tetrafluoride

In the present work, we have studied the two-frequency IRMPD of SiF$_{4}$ in a molecular jet. The dissociation eficiency and the enrichment factor were characterized with time of flight mass spectrommetry (TOF). The analysis of those main characteristics required the definition of estimators that disccounts the background signal.  
\section{Experimental approach}
\label{sec:1}

A scheme of the experimental setup is shown in figure \ref{fig:setup}. Two home-built TEA CO$_{2}$ laser were used as the excitation and dissociation sources. The excitation laser was tuned close to the Si-F stretching mode ($\nu_{3}$) to perform a vibrational excitation of the molecule. Furthermore, the dissociation laser was red shifted to a lower wave number from the $\nu_{3}$ vibrational mode. Both lasers have stable optical resonators and are focused through a 10 cm focal length ZnSe lens, in a collinear configuration.     

\begin{figure}[h]
	\includegraphics[width = 0.5\textwidth]{figures/dispositivo_2f_english.pdf}% Here is how to import EPS art
	\caption{\label{fig:setup} Experimental setup for the two frequency IRMPD over a molecular jet of SiF$_{4}$.}
\end{figure}

A sample of SiF$_{4}$ (99 \% Matheson) at a total pressure of 500 Torr is expanded through a pulsed valve (Parker Hannifin Corporation) into a stainless steel vacuum chamber, evacuated by a turbo-molecular pump (pump Brand). The average pressure in the molecular jet is estimated to be 1x10$^{-4}$ Torr \cite{bishop09}. Downstream, the molecular jet is crossed by the excitation and dissociation lasers and the ionization laser.

The ions generated are collected by an extractor potential into a Time of Flight Mass Spectrometer (Kore Technology) until they reach the ion detector. The spectrum signals obtained were recorded by an oscilloscope (Tektronix,DPO 7104 1GHz). We operate the CO$_{2}$ lasers at 1 Hz and the UV ionization laser at 2 Hz in order to discount the background on alternative shots. 

\subsection{Mass Spectrum analysis}
The irradiation of the molecular jet by the 266 nm UV laser produce a ionization but also a fragmentation of the SiF$_{4}$. Figure \ref{fig:spec_uv} show a typical mass spectrum of the sample irradiated only by the UV laser. This pattern has no peak in the 104 mass, which implies that it has no parent. Also it can be seen that the highest peak corresponds to the [SiF$^{+}$] ion so the amplitude of the peak is used as an estimator of the concentration.     
\begin{equation}
\text{SiF$_{4}$} \xrightarrow{h \nu_{uv}}
\begin{cases}
\text{SiF$_{3}$$^{+}$} + \text{F} + e^{-} \\
\text{SiF$_{2}$$^{+}$} + \text{2F} + e^{-} \\
\text{SiF$_{1}$$^{+}$} + \text{3F} + e^{-} \\
\text{Si$^{+}$} + \text{4F} + e^{-}
\end{cases}
\end{equation}

\begin{figure}[h]
	\includegraphics[width =1\linewidth]{figures/sp_uv_ir.pdf}
	\caption{\label{fig:spec_uv} SiF$_{4}$ mass spectrum obtained with 266 nm multi-photon ionization.}
\end{figure}

The molecular jet is first irradiated by the IR radiation that performs the following dissociation probability ($f^{i}$) for each isotope specie,

\begin{equation}
\text{SiF$_{4}$} \xrightarrow{\,\,\, h\nu_{\text{IR}} \,\,\,}
\begin{cases}
[^{28}\text{SiF$_{3}$}] = f^{28}[\text{SiF$_{4}$} ] \\
[^{29}\text{SiF$_{3}$}] = f^{29}[\text{SiF$_{4}$} ] \\
[^{30}\text{SiF$_{3}$}] = f^{30}[\text{SiF$_{4}$} ]
\end{cases}
\end{equation}

After this fragmentation, the sample is irradiated with the UV laser that produce a ionization but also a new fragmentation. 
\begin{equation}
\text{$^{j}$SiF$_{4}$} \xrightarrow{h \nu_{IR}} \text{$^{j}$SiF$_{3}$+ F} \xrightarrow{h \nu_{UV}} \text{$^{j}$SiF$^{+}$+ 3F}  
\end{equation}

Summarizing, the total signal (I$^{T}$) in the SiF$^{+}$ is a combination of both process.
\begin{equation}
I_{47}^{T} = K \text{p}_{\text{SiF$_{3}$}}^{\text{SiF$^{+}$}} f^{28}[\text{SiF$_{4}$}] + \text{K} \text{p}_{\text{SiF$_{4}$}}^{\text{SiF$^{+}$}} (1-f^{28})[\text{SiF$_{4}$}]
\end{equation}

From the ratio between the total mass spectrum and the UV spectrum an estimator proportional to $f^{i}$ is defined. The $\alpha$ estimator is the proportional to the dissociation yield,
\begin{equation}
\alpha_{j} = 
\begin{cases}
\alpha_{47} = (q-1)f^{28} \\
\alpha_{48} = (q-1)f^{29} \\
\alpha_{49} = (q-1)f^{30}
\end{cases}
\end{equation}

In the same way is possible to define the isotope selectivity estimator $\beta$,
\begin{equation}
\beta_{k} = 
\begin{cases}
\beta_{29} = \frac{\alpha_{47}}{\alpha_{48}} = \frac{f^{28}}{f^{29}} \\
\beta_{30} = \frac{\alpha_{47}}{\alpha_{49}} = \frac{f^{28}}{f^{30}}
\end{cases}
\end{equation}

\section{Results and discussion}
\subsection{Delay time laser dependence}

\subsection{Laser fluence dependence}
In figure \ref{fig:alpha_phi_excitation} we show the $\alpha^{2f}_{47}$ estimator's dependency with the dissociation laser fluence ($\Phi_{D}$). The experimental data is fitted by a power law function which is shown in orange dashed line. The exponent obtained is $a = 0.49\pm 0.02$, in good agreement with previous works [CITA GRUPO].  

\begin{figure}[h]
	\centering
	\includegraphics[width = 1\linewidth]{figures/alpha_47_phi_bombeo.pdf}
	\caption{\label{fig:alpha_phi_excitation} Dissociation yield estimator as a function of the dissociation laser fluence.}
\end{figure}

Although the power law function fits the experimental data points, we cannot be able to reach a saturation of the dissociation estimator.  

\subsection{Laser wavenumber dependence}
In general, the IRMPD dissociation probability shows a red shifted compare with normal modes oscillation frequency; mainly due to the vibrational anharmonicity. 

\begin{figure}[h]
	\centering
	\includegraphics[width = 1\linewidth]{figures/alfa_47_nu_bombeo.pdf}
	\caption{\label{fig:alfa_nu_bombeo} Dependence of the $\alpha_{47}$ estimator on the excitation laser frequency (blue dots) compare with the linear absorption IR spectrum of SiF$_{4}$.}
\end{figure}

Figure \ref{fig:alfa_nu_diso} show the $\alpha_{47}^{2f}$ estimator as a function of the dissociation laser wavenumber superimposed with the IR linear absorption spectrum. The fluences of the excitation and dissociation lasers were 50 J/cm$^{2}$ and 60 J/cm$^{2}$, and the delay was fixed at 1$\mu s$. As can be seen, the dissociation yield presents a resonance close to 980 cm$^{-1}$ which is almost 50 cm$^{-1}$ red-shifted compared to the IR linear absoprtion spectrum. This result is an indicative that the SiF$_{4}$ has been excited to the vibrational quasicontinuum by the excitation laser.  

As a rule of thumb we can assume a linear anharmonicity of the vibrational levels, founded on the Morse potencial. In the SiF$_{4}$ case, the anharmonicity constant is $\chi_{e}\nu_{3} \simeq 5 \text{cm}^{-1}$, so we can estimate that the molecule has been excited up to the $\nu_{SF} = 10$. 

\begin{figure}[h]
	\centering
	\includegraphics[width = 1\linewidth]{figures/alfa_47_nu_diso.pdf}
	\caption{\label{fig:alfa_nu_diso} Dependence of the $\alpha_{47}$ estimator on the dissociation laser frequency (blue dots) together with a gaussian distribution fit (orange dashed line), also a linear absorption IR spectrum of the $\nu_{3}$ can be seen.}
\end{figure}

\subsection{Enrichment factor}
The enrichment factor $\beta_{30}$ is plotted against the excitation and dissociation wavenumber, in figure \ref{fig:beta_spec}. The factor shows a resonance as a function of $\nu_{e}$ with an slightly red shift compare with Si-F stretching normal mode. On the other side, the resonance is shifted almost 50 cm$^{-1}$ in terms of the dissociation laser wavenumber. 

\begin{figure}[h]
	\centering
	\includegraphics[width = 1\linewidth]{figures/beta_30_nu_bombeo.pdf}
	\includegraphics[width = 1\linewidth]{figures/beta_30_nu_diso.pdf}
	\caption{\label{fig:beta_spec} Isotope enrichment factor as a function of lasers wavenumber, excitation and dissociation. Both are compared with the IR linear absorption spectrum of SiF$_{4}$.}
\end{figure}

\section{Conclusions}
	% BibTeX users please use
	% \bibliographystyle{}
	% \bibliography{}
	%
	% Non-BibTeX users please use

\begin{thebibliography}{}
	%
	% and use \bibitem to create references.
	%
	\bibitem{RefJ}
	% Format for Journal Reference
	Author, Journal \textbf{Volume,} (year) page numbers.
	% Format for books
	\bibitem{RefB}
	Author, \textit{Book title} (Publisher, place year) page numbers
	% etc
\end{thebibliography}
	
	
\end{document}

% end of file template.tex
